\setlength{\footskip}{8mm}

\chapter{RELATED WORK}
AIT Match draws inspiration from various online dating and social networking platforms. Mainstream dating apps like \citeauthor{tinder2024}, \citeauthor{bumble2024}, and \citeauthor{hinge2024} cater to a wide audience and utilize appearance-based matching systems. While effective for casual connections, they lack features suited to the academic and social needs of university communities, often missing out on fostering connections based on shared academic or professional interests.

Student-focused platforms like \citeauthor{friendsy2024} and \citeauthor{datamatch2024} attempt to bridge this gap by offering student-only dating experiences requiring university email registration. However, these platforms generally lack customization for individual institutions, limiting their effectiveness in facilitating connections in a specific academic setting like AIT.

Professional networking platforms such as \citeauthor{linkedin2024} and \citeauthor{facebookcampus2024} provide tools for academic and campus connections, but they are not tailored for romantic or deep social engagement. Similarly, broad social networks like \citeauthor{facebook2024} serve diverse users and interests, making it difficult to foster close, university-centered interactions within an academic environment. 

Moderation-focused apps like \citeauthor{coffeemeetsbagel2024} emphasize authenticity and safe interactions but do not offer university-verified spaces or prioritize academic connections. These limitations underscore the need for AIT Match, which combines the safety features of mainstream apps with the social and academic engagement crucial in a university community. Fig. \ref{fig:datingapps} illustrates the related applications.\newline

\begin{figure}[h]
    \centering % Center the image
    \captionsetup{justification=centering, singlelinecheck=false, labelsep=space} % Set up caption alignment
    \includegraphics[width=5.7in]{figures/datingapps.png} % Increased image width
    \caption{Related dating applications including Tinder, Bumble, Hinge, Friendsy, Datamatch, LinkedIn, Facebook, and Coffee Meets Bagels.}
    \label{fig:datingapps}
\end{figure}



% \begin{figure}
% \caption{Doge.}
% \centerline{\includegraphics[width=3in]{figures/doge.jpeg}}
% \label{fig:doge}
% \small{\textit{Note.} Additional notes goes here.}
% \end{figure}

% \section{Heading, Level 2}

% This section presents some guidelines on how to create and format tables and figures following the APA Style with some examples. Every table and figure should serve a purpose. A table or figure can be referred to in the text by its number [e.g., As shown in Table \ref{tab:dense}…, as can be seen in the results of the testing (see Figure \ref{fig:doge})].  Avoid writing “the table above” or “the figure below” as the position of a figure or table might change during the writing process.

% Tables and figures can be generated in different ways using many programs.  Table \ref{tab:dense} presents the format of a table following the APA style. Align all tables and figures with the left margin and place a table or figure after a paragraph where it is first mentioned. Separate the paragraph and the table or figure title by a double-spaced blank line. Titles should be brief, clear, and explanatory. 

% Repeat the column headings on the second page of the table (see Table \ref{tab:dense}). Separate this paragraph from the table by a double-spaced line. Tables and figures can be placed at the start or end of a page. Fit the table or figure between the margins and in one page.

% As there is very little space left for the table on this page, present the table on the next page.  You can add more content in this section.  The description should be as close to the table or figure as possible.
% (There should be one blank double-spaced line between the last line of the paragraph and the table or figure number, and between the table / figure number and the title.)

% You can cite stuff in references.bib like this \citep{doge}. $y$ is as follows.

% \begin{equation}
%     y = mx+b
% \end{equation}

% \section{Heading, Level 2}

% Add a short introductory sentence/s here.

% \subsection{Heading, Level 3}

% Start your paragraph here. Table 2.2 presents a sample of a qualitative table with variable descriptions. Separate the paragraph and the table or figure title by a double-spaced blank line. Titles should be brief, clear, and explanatory.  Check the Language Center website for more examples.

% \subsection{Heading, Level 3}

% (1 space between the last line of this section and the next Level 2 heading)

% \section{Heading, Level 2}
% As for figures, the figure title should also be written in italics below the figure number (in bold) separated by a double-spaced blank line as shown in Figure 2.1.  The size and density of the elements in a figure must be considered when deciding on the font size and spacing.   Continue with the paragraph here.

% Continue with the paragraph here.  The table or figure should be as close to the description as possible or when it is first mentioned. Fit the tables and figures between the margins.
% (There should be one blank double-spaced line between the previous paragraph and the figure number, and between the figure number and the title.)

% \begin{table}[]
% \caption{An example table in latex.}
% \begin{center}
% \begin{tabular}{l l}
% \hline
%     Methods & Metric\\ \hline
% Method A      & 153.3                \\ 
% Method B & 2.4                  \\ \hline
% \end{tabular}
% \label{tab:dense}
% \end{center}
% \small{\textit{Note.} Add notes here.}
% \end{table}
