\setlength{\footskip}{8mm}

\chapter{CONCLUSION}
The AIT Match platform represents a significant advancement in creating a secure, personalized social network designed specifically for university students. By integrating a robust back-end stack that includes Rails, Ruby, PostgreSQL, HTML, CSS, JavaScript, and Docker, this platform enables efficient and scalable real-time interactions. Core features such as personalized profiles, academic filters, interest-based connections, and real-time messaging cater directly to the unique needs of students at the Asian Institute of Technology (AIT), facilitating meaningful connections in a trusted and secure environment.

AIT Match transcends conventional dating platforms by emphasizing both academic and social engagement, offering users a space to connect with peers who share similar academic programs, interests, or career aspirations. The platform’s matching algorithm, privacy-focused infrastructure, and verified user profiles ensure that interactions are safe and authentic, fostering a strong sense of community within AIT. Users are required to log in with their university email, adding an extra layer of security by verifying users’ affiliation with AIT.

The platform also includes a report system within the admin framework, allowing users to report inappropriate behavior by submitting a report form that includes the reason for reporting. Administrators have the authority to review these reports and, if necessary, delete the reported profiles from the platform, enhancing security and ensuring a respectful environment by removing problematic users.

The platform’s deployment on AIT’s CSIM (Computer Science \& Information Management) server leverages institutional support for stable and secure access, while Docker and DockerHub facilitate scalability and maintainability for future development. GitHub serves as a centralized repository, supporting version control and collaborative development management. Overall, AIT Match not only demonstrates technical excellence but also showcases the potential of targeted social networks to enhance student life through friendship, collaboration, and academic connections, setting a precedent for similar platforms in educational environments.

